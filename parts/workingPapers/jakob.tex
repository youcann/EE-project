 \subsection{Syngas (Jakob)}
 
 \begin{xtabular}{|p{3.8cm}|p{8.3cm}|p{4.2cm}|}
 	\vspace*{-1.25\baselineskip}\subsubsection{Topic Title}
 	& 
 	This text compares different raw materials, such as wood/crops, algae and food waste, for their possible use in the biofuel production with the means of the Bioliq-pyrolysis process. Therefore, the Energy Return on Investment (EROI) was postulated and taken in account for a method of comparison.
 	& 
 	\\
 	\vspace*{-1.25\baselineskip}\subsubsection{Introduction}
 	& 
 	The production of syngas with the means of the Bioliq-process is divided into two main steps:
 	\begin{enumerate}
 		\item Quick pyrolysis of biomass with the help of hot sand at \SI{500}{\degreeCelsius} for 3-5 seconds.
 		\item Transporting of the produced char and condensate for further processing to the central entrained flow gasifier unit
 	\end{enumerate}
 	&
 	\mydef{Syngas: Fuel gas consisting primarily of hydrogen and carbon monoxide} 
 	\\
 	%heading
 	
 	&
 	%main
 	To avoid the cost and energy intensive transport of raw material, the first step is taken place decentralized in small pyrolysis facilities that cover a reasonable range.
 	&
 	%ref
 	
 	\\
 	%heading
 	
 	&
 	%main
 	For pyrolysis a raw material biomass is necessary and therefore studied in this text. For a suitable raw material, the following criteria should be met:
 	\begin{itemize}
 		\item No conflict with food and feed production
 		\item Higher or same EROI than wood/straw (i.e. a high annual yield of biomass with a specific calorific value taken in account the energy consumption of growing and harvest)
 	\end{itemize}
 	&
 	%ref
 	
 	\\
 	%heading
 	Different Types of Biomass
 	&
 	%main
 	Biomass in the context of this work, is regarded as the mass that is produced by and/or contains different organisms and can be used as a raw material for the pyrolysis.
 	&
 	%ref
 	
 	\\
 	%heading
	Wood and Crops
 	&
 	%main
 	The 'traditional' used raw material for the pyrolysis in the Bioliq-plant of KIT is consisting of wood (fast-growing like willow or poplar) and crops like wheat or corn. 
 	&
 	%ref
 	\mydef{KIT: Karlsruhe Institute of Technology}
 	\\
 	%heading
 	Algae
 	&
 	%main
 	Algae are produced in different types of reactors, for example flat plates or open ponds regarding their specific properties and the required conditions (e.g. sterility, purification, process attributes). In this text a fermentation of algae in open pond reactors is considered due to the following reasons:
 	&
 	%ref
 	\mydef{Fermentation: Production of Biomass using organic sources}
 	\\
 	%heading

 	&
 	%main
 	\begin{itemize}
 		\item Algae are cheap to produce
 		\item They need no special media other than water, $\text{CO}_2$ and sunlight.
 		\item The water that is used may even be brackish and has not to be treated in a special way.
 	\end{itemize}
 	&
 	%ref
 	
 	\\
 	%heading
 	Food Waste
 	&
 	%main
 	Food waste is produced either during the production of food by the food industry or later in retail and in the households. The amount of waste in Germany is regardably high with about 11 million tons of food per year.
 	&
 	%ref
 	
 	\\
 	\vspace*{-1.25\baselineskip}\subsubsection{Findings}
 	& 
 	Because of the decentralized positions of the pyrolysis facilities near to their source of raw materials it was decided to compare the different EROIs by space (hectare) and time (year). The results are displayed in the following table:
 	&
 	\\
 	%heading
 	
 	&
 	%main
 	{
 	\tiny
 	\begin{tabularx}{8cm}{X|ccc}
 	                             	   & Food Waste & Algae  & Wood/Crops        \\
 	    \hline
 		Calorific values [MJ/kg]       & 4.1        & 13.5   & 17.5              \\[8ex]
 		Yield [t/(ha*a)]               & 2.5        & 27     & 7\dots14          \\[8ex]
 		Energy Consumption [GJ/(ha*a)] & -          & 1.85   & 1.85              \\[8ex]
 		EROI [GJ/(ha*a)]               & 10.5       & 362.65 & 120.65\dots243.15 \\
 	\end{tabularx}
	}
 	&
 	%ref
 	
 	\\
 	%heading
 	Estimation of Usable Energy in Food Waste
 	&
 	%main
 	The calculation of EROI in the food waste was done by a few estimations. First, the calorific value had to be set. Naturally this value has a broad range due to the many different forms of food. This led to a reasonable value somewhere in between a pizza and some cauliflower amounting approx. 4100 kJ/kg.
 	&
 	%ref
 	\mydef{Fddb: Keywords: Pizza, Cauliflower}
 	\\
 	%heading
 	
 	&
 	%main
 	For the yield of food waste, its production of 11 Million tons was compared to built-up land, consisting of residential, commercial and industrial buildings and transport infrastructure, that adds up to 4,309,700 hectares. The yield therefore lies approximately at 2.5 tons/(ha*a).
 	These calculations end up with the value of 10.5 GJ/(ha*a) of usable energy in cities and dense settlements.
 	&
 	%ref
 	\mydef{Bfn: Land use in Germany}
 	\\
 	%heading
 	Usable Energy in Algae and in Wood/Crops
 	&
 	%main
 	For algae, especially for those containing a high amount of oils, a rather high calorific value of 12.5-14.5 MJ/kg can be found. According to the used sources the yield in an open pond reactor approximates 18.3 to 36.6 tons/(ha*a). For further calculations an averaged value of 27 tons/(ha*a) was taken in account.
 	&
 	%ref
 	Sciencedirect (I): Microalgae Biomass: A Renewable Source of Energy
 	\newline
 	\mydef{Farm-energy:	Algae for biofuel production}
 	\\
 	%heading
 	
 	&
 	%main
 	The energy consumption during the growth and the harvest of algae is added up by the replacement of water, the energy used by a pump to circulate the ponds and the heat that is needed for drying the harvested algae. It is considered, that the consumption of Energy equals that of agriculture, which is approx. 1,85 GJ/(ha*a). In both cases this value must be taken off the amount of Energy contented in the dry Biomass to get the EROI.
 	&
 	%ref
 	Landwirtschaftkammer: Energieffizienz in der Landwirtschaft (efficiency of energy in agriculture)
 	\\
 	%heading
 	
 	&
 	%main
 	For the use of wood and crops the calorific value in literature is stated with approx. 17.5 MJ/kg with only little differences in between wood and crops set especially by their amount of oil. Also, the yield equals in wood and crops, considering that the grain is used as well as the straw, and a value of 7 to 14 tons/(ha*a) can be found. 
 	&
 	%ref
 	Sciencedirect (II): Biomass yield and energy value of some fast-growing multipurpose trees in Nigeria
 	\newline
 	Ourworldindata:Long term wheat yields
 	\\
 	%heading
 	
 	&
 	%main
 	For both, algae and wood/crops the calorific value is multiplied with the yield and the consumed energy during culture is subtracted to obtain the maximum amount of usable energy, which are listed in the table above. 
 	&
 	%ref
 	
 	\\
 	\vspace*{-1.25\baselineskip}\subsubsection{Problems encountered}
 	& 
 	At first it has been the main idea to compare different types of wood and grains, in order to find the best suitable option for a usable biomass. After a bit of research, it occurred to be that there is not much difference in between the calorific value of those 'traditional used' plants as straw and wood. 
 	&
 	Tfz Bayen: Heizwerttabelle (tabled calorific values) 
 	\\
 	%heading
 	
 	&
 	%main
 	Although it might be worth to even compare those regarding the yield per space and time, the decision was made to consider them as the basic choice for pyrolysis in the context of biofuel production and look for other comparable types of biomass, as algae and food waste. 
 	&
 	%ref
 	
 	\\
 	\vspace*{-1.25\baselineskip}\subsubsection{Discussion}
 	& 
 	Regarding the high EROI of algae and the fact that there is the possibility of using space for cultivation that is not arable for food production, algae should be considered as a choice for use in the Bioliq® pyrolysis process. Even more, the use of brackish water and the absence of fertilizers, insecticides and pesticides is another big advantage justifying this decision. A big disadvantage is the need for sunlight, which especially beginning within the temperate zones declines.
 	&
 	\\
 	%heading
 	
 	&
 	%main
 	However, the usage of wood and crops is a well-established method. Nonetheless it is in a steady competition to agriculture for food and animal feed and should therefore be reconsidered.
 	&
 	%ref
 	
 	\\
 	%heading
 	
 	&
 	%main
 	One should also consider the use of food waste especially in bigger cities, for there is no chance of doing agriculture as well. But nevertheless, food waste is already composted and used as a fertilizer for farming, so there is no necessity to urgently find a way processing it. On contrary, the focus should rather lie on avoiding the production of more food waste.
 	&
 	%ref
 	
 	\\
 	\vspace*{-1.25\baselineskip}\subsubsection{Current Developments}
 	& 
 	In the last decade there is an upcoming research for the pyrolysis of algae biomass. It is described as highly efficient with a high yield and in the typical process parameters. The algae cultivation is also used as a wastewater treatment step as well. 
 	&
 	Frontiers: Pyrolysis of algal biomass obtained from high-rate algae ponds applied to wastewater treatment
 	\\
 	\vspace*{-1.25\baselineskip}\subsubsection{Conclusion}
 	& 
 	Due to their high EROI, their undemanding cultivation on non-arable land and the possibility to connect wastewater treatment with energy harvest, algae should be the medium of choice for use in the Bioliq-pyrolysis process.
 	&
 	\\
 	\vspace*{-1.25\baselineskip}\subsubsection{Recommendation}
 	& 
 	Both, the production of algae in open ponds, as well as the processing of biomass with the means of pyrolysis are well-established. It is therefore recommended to use both principles and combine them for a possible production of biofuel within the Bioliq-process.
 	&
 	\\
 	\vspace*{-1.25\baselineskip}\subsubsection{Personal Comments}
 	& 
 	To determine the calorific value of food waste, I first thought of taking the value of pizza. But than again, no one is throwing a pizza into the garbage. Probably spinach or cauliflower is thrown away more often. Therefore, I decided to take a value in between those.
 	&
 	\\
 	\hline
 \end{xtabular}